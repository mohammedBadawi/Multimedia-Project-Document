\chapter{Conclusion}
CBIR  is  a  fast-developing  technology  with considerable  potential  in  digital  libraries, architectural  and  engineering  design,  crime prevention,  historical  research  and  medicine. Nevertheless,  the  effectiveness  of  current  CBIR systems  is  inherently  limited  because  they  most effective operate  at  the  primitive  feature  level. Furthermore,  the  technology  still  lacks  maturity, and  is  not  widely  used  on  a  significant  scale. Consequently,  examine  examines  exclusive techniques  used  in  CBIR  systems.  The  examine reviewed  numerous  literatures  that  are  related  to CBIR.  The  examine  located out  that  there  are  3 basic  capabilities  that  can  be  extracted  in  CBIR. These  consist of  shade,  texture  and  shape.  The examine  additionally  discovered that  each  of  those  capabilities has  exclusive  extraction  techniques.  For  instance, shade  can  be  extracted  in  images  using  shade histogram,  geometric  moments,  shade  area     Olaleke et al.; AJRCOS, 3(2): 1-15, 2019; Article no.AJRCOS.48235    14  and  shade  moments.  The  examine  discovered  the strengths  and  weaknesses  of  each  of  those techniques.  For  instance,  the  shade  area technique is straightforward to put in force however it isn't uniform while  the  shade  histogram  is  faster  and  greater green than  other  shade  extraction  techniques.  It can  however  be  identical  for  two  images  with exclusive colours. The examine additionally famous that  the GLCM,  Tamura,  Fourier  transform,  Ranklet transform  and  discrete  wavelets  are  ordinary examples  of  textural  extraction  techniques. Similarly,  the  edge  technique,  Fourier  descriptors and  Zernike  technique  were  the  shape  extraction techniques discovered  on this examine. Furthermore, the examine  investigated  the techniques  for  computing the  similarity  between  a  query  image  and  the images  in  the  database.  The  result  of  the  examine showed  that  examples  of  similarity  measures utilized in CBIR consist of sum of absolute difference, sum  of  the  squared  differences  of  absolute values and metropolis block distance.  In latest times, there's no popular leap forward in CBIR regardless of the numerous techniques and equipment developed  to  formulate  and  execute  queries  in large  databases  based  on  their  visual  contents. Hence,  future  works  should  be  tailored  closer to the  development  of  CBIR  systems  that  will solve the trouble of semantic hole in CBIR.
